\chapter{Model}
	\section{EIRENE 1D}
		\begin{itemize}
			\item What is Eirene\\
				\begin{itemize}
					\item Monte Carlo Simulation of plasma particles
					\item solving kinetic boltzmann equations to propagate particles
					\item using plasma chemical reaction rates for interactions
					\item Plasma Profiles
				\end{itemize}
			\item Why 1DModel instead of full Eirene\\
				\begin{itemize}
					\item Reduction of run time of each simulation step.
					\item Removes geometric parameters from input parameter set.
				\end{itemize}
			\item Additional assumptions\\
				\begin{itemize}
					\item $T_I =T_e$ reasonable assumption to further reduce dimensionality of input.
				\end{itemize}
		\end{itemize}
		% Mitja Beckers
		%\subsection{Point source}
		%\subsection{Volume source}
		\subsection{Kinetic Boltzmann Equations}
	\section{Plasma Profiles}
		For the inputs of Eirene plasma profiles are needed, that can be dynamically provided by other algorithms like SOLPS\todo{Citation needed} or EMC3\todo{Citation needed}. Central piece of this work is to investigate if a substitute function can be found for the full range of possible plasma profiles by using big data methods. One can ascertain the physical limits of the parameters constituting the plasma profiles from table \ref{Par_Bounds}. These limits are based on different phenomenons in plasma physics, which can be 
		\todo{Insert Plasma Profiles and table from RedMod Workshop}
	\section{Choice of sampling set}
		Since the parameter space is high dimensional, the training points have not been selected randomly. According to \todo{Add reference} randomly sampled points might form clusters, which could skew the training towards a subsection of the parameter space. To avoid this a low discrepancy sequence, namely the Sobol sequence, from which the training points are sampled has been chosen.
		%\missingfigure{Clustering of points sampled from equal distribution}
		\subsection{Sobol Sequence}
			\begin{itemize}
				\item short overview
				\item formula
				\item advantages
			\end{itemize}
			The sobol sequence was first invented by the mathematician Ilya M. Sobol\todo{Add citation} 
			\subsubsection{Low-discrepency sequences}
				\todo{More details on advantages of low-discrepency sequences.}