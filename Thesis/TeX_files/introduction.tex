\chapter*{Abstract}

\chapter{Introduction}
	The design process for a facility like a fusion power plant takes into account a manifold of aspects. Thereunder a cost analysis for the fusion device. To make an estimate of the cost analysis one has to consider the lifetime of machine parts. Most prominently the divertor and first wall suffer from shortened life spans due to erosion. Which is partly due to neutral particle induced sputtering.\\
	To include considerations like these in the design of a power plant one uses so called systems codes like PROCESS\cite{process}. These codes focus on optimizing design parameters of large scale systems like power plants, which consinst of many smaller subsystems. Due to the amount of subsystems the need arises to simplify simulations to achieve reasonable run times for systems codes. The following work is concerned with deducing a fast surrogate in place of a simulation for the sputtering rate of a fusion device component.\\
	The following chapter gives a brief overview of the motivating applications while also introducing the concept of reduced model approaches via machine learning algorithms. Furthermore it considers which methods are most applicable in the given situation.\\
	
	\section{PROCESS Systems Code}
	%TODO Elaborate on difference in point of view: Engineering + marketing vs. Scientific Simulation
	The systems code PROCESS\cite{process} is concerned with the 
	
	%TODO 
	%TODO Example 
	
	\section{Fusion Devices}
	
	%TODO Picture of fusion device
	%TODO Elaborate on difficulty of fusion device longevity
	
	\section{Reduced Model Approaches}
